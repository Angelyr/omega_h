\documentclass{article}

\usepackage{hyperref}
\usepackage{listings}
\lstset{
  basicstyle=\ttfamily,
  columns=fullflexible,
  keepspaces=true,
  numbers=left,
  xleftmargin=2em
}

\title{The Omega\_h Users Manual}
\author{Dan Ibanez\\
Sandia National Laboratories\\
daibane@sandia.gov}

\begin{document}

\maketitle

\section{Obtaining}

Omega\_h is developed and distributed via GitHub,
a popular software hosting platform based on the
Git distributed version control system.

\url{https://github.com/ibaned/omega_h}

The most common way to obtain Omega\_h is to
use Git to clone the repository and automatically
check out the \texttt{master} branch:

\begin{lstlisting}[language=bash]
git clone git@github.com:ibaned/omega_h.git
\end{lstlisting}

\section{Compiling}

\subsection{Operating System}

Omega\_h currently only supports POSIX-like operating systems.
It has only been tested on Linux, Mac OS X, and certain POSIX-like
supercomputer kernels.

\subsection{CMake}

Omega\_h's compilation process is controlled by the CMake
build system.
In order to build Omega\_h, one should install a recent
version of CMake, with version 3.0.1 being the minimum
acceptable version.
CMake is available in most Linux package managers
or from their website:

\url{https://cmake.org/download/}

It is also recommended that users adopt the CMake build
system for their own project, in order to take advantage
of the metadata that Omega\_h outputs when it compiles
and installs, which is readable by other CMake projects.

\subsection{Compiler}

Omega\_h is written in the C++11 standard of C++,
so a compiler with complete support for that standard
is needed.
Omega\_h's CMake files will accept the GCC, Clang,
and Intel compilers, and support can be added
fairly easily for other compilers upon request.

\subsection{Dependencies}

All dependencies of Omega\_h are optional,
meaning that one can compile it by itself and obtain a fairly
functional code for mesh adaptation, although
it will not have parallel features yet.
Optional dependencies of Omega\_h are:

\begin{enumerate}

\item Zlib: This widely used and installed C library implements
efficient data compression algorithms.
Omega\_h uses it compress its own `.osh` file format and
VTK's `.vtu` files.

\item MPI: The Message Passing Interface is a standard
defining (at least) a C library that enables multi-process parallelism.
This is required if you want to use multi-process parallelism
in Omega\_h.
The two good open-source implementations that Omega\_h is known
to work with are MPICH and OpenMPI, and we recommend
MPICH for its support of the latest MPI standard and its
cleaner memory management.

\item Kokkos: This C++11 library implements shared-memory parallelism
constructs and allows Omega\_h to (mostly) not worry about the details
of OpenMP and CUDA.
It is required if you want to use shared-memory parallelism in Omega\_h.
Kokkos can be obtained as part of Trilinos, and Trilinos can be configured
to compile and install only Kokkos.

\item libMeshb: This C library implements the `.mesh` and `.meshb`
file formats used by INRIA, NASA, and others.
It is required to read and write `.meshb` files from Omega\_h.
Note that currently Omega\_h follows a particular convention in what
those files are expected to contain, namely elements, vertices,
and sides on the boundary.

\item EGADS: This C API wraps over OpenCASCADE in a human-manageable way.
It is required if you want Omega\_h to snap new vertices to geometry.
Note that classification in Omega\_h should match the numbering
of geometric entities in EGADS.

\end{enumerate}

Each of these dependencies

\subsection{Options}

\section{Usage}

\subsection{Utility Programs}

\begin{enumerate}
\item texttt{msh2osh}
\item texttt{osh2vtk}
\item texttt{oshdiff}
\item texttt{vtkdiff}
\item texttt{osh\_box}
\item texttt{osh\_part}
\item texttt{osh\_scale}
\item texttt{meshb2osh}
\item texttt{osh2meshb}
\end{enumerate}

\subsection{Header and Library}

\subsection{Via CMake}

\subsection{The Omega\_h Namespace}

\section{Read and Write: The Array Classes}

\section{The Mesh: A (Mostly) Immutable Cache}

\end{document}
