\documentclass{article}

\usepackage{hyperref}
\usepackage{listings}
\lstset{
  basicstyle=\ttfamily,
  columns=fullflexible,
  keepspaces=true,
  xleftmargin=2em
}

\title{The Omega\_h Users Manual}
\author{Dan Ibanez\\
Sandia National Laboratories\\
daibane@sandia.gov}

\begin{document}

\maketitle

\section{Purpose}

The purpose of Omega\_h is to adapt triangle and tetrahedral
meshes to conform to a given metric field,
effectively using combinations of MPI and shared-memory parallelism,
and providing good results as defined by element quality and
edge length measures.

\section{Obtaining}

Omega\_h is developed and distributed via GitHub,
a popular software hosting platform based on the
Git distributed version control system.

\url{https://github.com/SNLComputation/omega_h}

The most common way to obtain Omega\_h is to
use Git to clone the repository and automatically
check out the \texttt{master} branch:

\begin{lstlisting}[language=bash]
git clone git@github.com:SNLComputation/omega_h.git
\end{lstlisting}

\section{Compiling}

\subsection{Operating System}

Omega\_h currently only supports POSIX-like operating systems.
It has only been tested on Linux, Mac OS X, and certain POSIX-like
supercomputer kernels.

\subsection{CMake}

Omega\_h's compilation process is controlled by the CMake
build system.
In order to build Omega\_h, one should install a recent
version of CMake, with version 3.0.0 being the minimum
acceptable version.
CMake is available in most Linux package managers
or from their website:

\url{https://cmake.org/download/}

It is also recommended that users adopt the CMake build
system for their own project, in order to take advantage
of the metadata that Omega\_h outputs when it compiles
and installs, which is readable by other CMake projects.

\subsection{Compiler}

Omega\_h is written in the C++11 standard of C++,
so a compiler with complete support for that standard
is needed.
Omega\_h's CMake files will accept the GCC, Clang,
and Intel compilers, and support can be added
fairly easily for other compilers upon request.

\subsection{Dependencies}

All dependencies of Omega\_h are optional,
meaning that one can compile it by itself and obtain a fairly
functional code for mesh adaptation, although
it will not have parallel features yet.
Optional dependencies of Omega\_h are:

\begin{enumerate}

\item Zlib: This widely used and installed C library implements
efficient data compression algorithms.
Omega\_h uses it compress its own \texttt{.osh} file format and
VTK's \texttt{.vtu} files.

\item MPI: The Message Passing Interface is a standard
defining (at least) a C library that enables multi-process parallelism.
This is required if you want to use multi-process parallelism
in Omega\_h.
The two good open-source implementations that Omega\_h is known
to work with are MPICH and OpenMPI, and we recommend
MPICH for its support of the latest MPI standard and its
cleaner memory management.
In order to use this, we typically just use the MPI compiler wrapper
\texttt{mpicxx} as the \texttt{CMAKE\_CXX\_COMPILER}.

\item Kokkos: This C++11 library implements shared-memory parallelism
constructs and allows Omega\_h to (mostly) not worry about the details
of OpenMP and CUDA.
It is required if you want to use shared-memory parallelism in Omega\_h.
Kokkos can be obtained as part of Trilinos, and Trilinos can be configured
to compile and install only Kokkos.
Enabling Kokkos provides two more options, \texttt{Omega\_h\_ENABLE\_CUDA}
and \texttt{Omega\_h\_ENABLE\_OpenMP}, which are the main on-node
parallelism backends that Kokkos uses for GPUs and multi-core CPUs,
respectively.

\item libMeshb: This C library implements the `.mesh` and `.meshb`
file formats used by INRIA, NASA, and others.
It is required to read and write `.meshb` files from Omega\_h.
Note that currently Omega\_h follows a particular convention in what
those files are expected to contain, namely elements, vertices,
and sides on the boundary.

\item EGADS: This C API wraps over OpenCASCADE in a human-manageable way.
It is required if you want Omega\_h to snap new vertices to geometry.
Note that classification in Omega\_h should match the numbering
of geometric entities in EGADS.

\end{enumerate}

Each of these dependencies is first controlled by a CMake boolean,
for example the dependence on Kokkos is controlled by the boolean
\texttt{Omega\_h\_ENABLE\_Kokkos}.
Dependencies which require finding libraries and/or headers have
an associated prefix path where those libraries and headers are installed,
for example \texttt{Kokkos\_PREFIX} indicates the location where
Kokkos was installed (in that particular case it is usually the
Trilinos installation location).
So to use Kokkos one may give the following two arguments to the
\texttt{cmake} command-line program:
\begin{lstlisting}
-DOmega_h_ENABLE_Kokkos:BOOL=ON \
-DKokkos_PREFIX:PATH=$HOME/install/Trilinos \
\end{lstlisting}

\subsection{Other Options}

\section{Usage}

\subsection{File Format}

Omega\_h implements I/O functionality for a custom mesh format.
The format stores parallel meshes using one file per partition,
and stores them all in a single directory which is typically named
with the \texttt{.osh} extension to indicate that it is conceptually
an Omega\_h ``file".
The directory may look something like this:
\begin{verbatim}
cube.osh/
cube.osh/nparts
cube.osh/version
cube.osh/0.osh
cube.osh/1.osh
cube.osh/2.osh
cube.osh/3.osh
\end{verbatim}
The file \texttt{cube.osh/2.osh} encodes, in a binary format, the
mesh partition stored in MPI rank 2.
As will be described in Section \ref{sec:rw}, Omega\_h stores most
information into arrays.
In order to write this information to file, byte swapping is first
applied to the entries as a way to ensure that information remains
consistent between little endian and big endian computers,
and then the array is compressed with Zlib (assuming the user has
compiled Omega\_h with Zlib).
These two steps ensure the file is both compact and lossless, i.e.
the mesh data when loaded from file will be bitwise identical to
the data that was written to file.

This format is a convenient way for users to manage meshes and attached
information, and to build an ecosystem of modular command-line tools
which act on said meshes.

It is recommended that users only parse this format using
the APIs provided by the Omega\_h library.

We do try to ensure the format is backwards-compatible, i.e.
a file tree written by one version of Omega\_h should remain readable
by all subsequent versions of Omega\_h.

\subsection{Utility Programs}

For some basic operations, mostly involving file format conversions,
Omega\_h will compile command-line programs that perform these operations.

\begin{enumerate}
\item \texttt{msh2osh}: Convert from Gmsh's native text format to the Omega\_h binary format.
\item \texttt{osh2vtk}: Convert from the Omega\_h binary format (tree) to a tree as follows:
\begin{verbatim}
given-name/
given-name/pieces.pvtu
given-name/pieces/
given-name/pieces/piece_0.vtu
given-name/pieces/piece_1.vtu
\end{verbatim}
Where \texttt{given-name} is the second argument provided to the program.
Each \texttt{.vtu} file represents one parallel partition, and opening the \texttt{.pvtu}
in Paraview (or another viewer) will render the entire parallel mesh.
\item \texttt{meshb2osh}, \texttt{osh2meshb}: these two programs become available
when the \texttt{libMeshb} dependency is enabled, and convert back and forth
between the Gamma mesh format (sometimes used by INRIA and NASA for CFD applications)
and (serial) Omega\_h file trees.
\item \texttt{oshdiff}: Modeled after the \texttt{exodiff} program from the Exodus II
project, this program compares two Omega\_h mesh files with configurable floating-point
tolerances.
\item \texttt{vtkdiff}: Modeled after the \texttt{exodiff} program from the Exodus II
project, this program compares two VTK file trees with configurable floating-point
tolerances. The trees must have been created by Omega\_h in the first place, due
to limitations in Omega\_h's VTK format reader.
\item \texttt{osh\_box}: Generates a structured simplex mesh of a parallelepiped domain,
by first constructing a quadrilateral or hexahedral grid and then subdividing said grid
into triangles or tetrahedra.
The parameters are the extents of the box and the number of elements that span
the length of the box along each axis.
\item \texttt{osh\_part}: Partitions an Omega\_h mesh. The input is a \texttt{.osh} file
tree with a certain number of parts, and the output is a \texttt{.osh} file tree with
a different number of parts.
An increase in the number of parts must multiply the number of parts by a power of two,
due to limitations of Omega\_h's Recursive Inertial Bisection implementation.
The number of parts may also be decreased, in which case certain consecutive parts will
be merged.
\item \texttt{osh\_scale}: Adapts a mesh such that it retains approximately the
same variation of resolution (metric gradient) and upon output contains approximately
the desired number of elements.
This can be convenient for convergence studies, to generate a series of meshes which
have prescribed element counts and whose resolution variation over the spatial domain
mirrors the variation in resolution of the initial starting mesh.
\end{enumerate}

\subsection{Header and Library}

In most cases, users should call the C++11 API of Omega\_h directly
from their own C++ code.
In terms of the build system, users need their code to include
the \texttt{Omega\_h.hpp} header file and link their own code
to the \texttt{libomega\_h.so} or \texttt{libomeg\_h.a} library.
In the case of static linking (\texttt{libomega\_h.a}), it is
also necessary to link to the final executable to all libraries
that Omega\_h depends on.
In the case of dynamic linking, \texttt{libomega\_h.so} is guaranteed
to already be linked to its dependencies via the \texttt{RPATH}
mechanism, so only \texttt{libomega\_h.so} needs to be linked
to the user application.

\subsection{Via CMake}

The most convenient way to handle the header file inclusion
and library linking to Omega\_h is to use the CMake build system
for the user's project as well.
Omega\_h uses CMake to export file

\texttt{\$\{CMAKE\_INSTALL\_PREFIX\}/lib/cmake/Omega\_hConfig.cmake}

which provides all the necessary include and link information
to CMake when CMake is configuring the user's project.
The way for the user's project to read that file is via
CMake's \texttt{find\_package} command:

\begin{lstlisting}
find_package(Omega_h 8.2.0 PATHS ${OMEGA_H_PREFIX})
\end{lstlisting}

Only the first argument (``Omega\_h") is really necessary.
The second argument specifies a minimum required version number.
CMake will only accept an Omega\_h installation that has
the same or higher version number, unless the major version number
is higher, because that indicates a change that breaks backwards
compatibility.
The last two arguments can be used to indicate that Omega\_h
is installed somewhere other than the standard system directories
(e.g. \texttt{/usr/local}).
If \texttt{find\_package} succeeds, subsequent CMake code has
access to a target called \texttt{Omega\_h::omega\_h}, which
represents all the information needed to link to Omega\_h,
and is used via \texttt{target\_link\_libraries} with a user's
library or executable target, like so:
\begin{lstlisting}
add_executable(my_simulation_code driver.cpp algebra.cpp)
target_link_libraries(my_simulation_code Omega_h::omega_h)
\end{lstlisting}

\subsection{The Omega\_h Namespace}

As a C++ library, Omega\_h has a responsibility not to pollute
the user's global namespace, whether it is the C++ namespace
or the preprocessor namespace.
As such, all its public C++ symbols will be contained in a C++
namespace called \texttt{Omega\_h}, and all preprocessor
symbols that it publicly defines have names beginning with
\texttt{OMEGA\_H}.
A relevant C++ language feature is the ability to
alias namespaces, i.e. users could do the following:
\begin{lstlisting}[language=C++]
namespace osh = Omega_h;
\end{lstlisting}
and in subsequent code Omega\_h symbols would be available
in the namespace \texttt{osh}.

\section{Public Symbols}

\subsection{Scalar Types}
\label{sec:scalars}

Omega\_h builds all its data structures out of arrays
of basic scalar types.
Over time it seems that a set of four basic types
is sufficient to form efficient data structures
for all its needs: one floating-point type
(\texttt{double}), and three integers types
of varying width (\texttt{int8\_t}, 
\texttt{int32\_t}, and \texttt{int64\_t}).
Omega\_h names these types \texttt{Real},
\texttt{I8}, \texttt{LO} (a.k.a. \texttt{I32}),
and \texttt{GO} (a.k.a. \texttt{I64}).
\texttt{LO} and \texttt{GO} stand for local
and global ordinals, terms borrowed from
the Trilinos project.
There is also \texttt{Int}, which is meant for
local small integer values, not as the entry
type for a large array.

\subsection{Read and Write: The Array Classes}
\label{sec:rw}

The vast majority of Omega\_h data is stored in
contiguous, dynamically allocated arrays.
These arrays are implemented as four templated
classes, defined by two binary properties.

First, Omega\_h gains certain benefits of functional programming
related to ``const-correctness" by having separate types
for arrays which are modifiable and those which are not.
All data starts as a modifiable array (``Write"), and as soon as possible
it is converted to a read-only object (``Read").
Ideally, there is only one modifiable array in existence at any
particular time: the array being created by the currently
executing code.
One of the more interesting benefits of this is the ability
to use shallow copies of data with confidence via the
Copy-On-Write convention, for example two mesh objects
may share the same coordinates array with confidence
that if one mesh decides to change its coordinates,
the other mesh will not be affected.

Second, Omega\_h is designed for efficient use of
GPUs, which have memory that is separate from
the memory attached to the CPU.
At the time of this writing, explicit control of where
memory resides and when to transfer it from the GPU
to the CPU is the most reliable way to get good
performance on GPU-accelerated systems.
In addition, due to the high cost of memory transfer,
the best performance is achieved by keeping data as
much as possible on the GPU and not moving it to the
CPU unless absolutely necessary.
Hence, each array type includes ``Host" classes,
which represent a copy of the array on the CPU.
These variants are the minority, most arrays in Omega\_h
are in GPU memory when it is compiled with CUDA support.

In total, the four classes are \texttt{Read}, \texttt{Write},
\texttt{HostRead}, and \texttt{HostWrite}.
Each of these four classes, in turn, is templated on
a scalar type, and explicitly instantiated for
each of the four scalar types described in Section
\ref{sec:scalars}, so for example \texttt{Read<Real>}.
Finally, some convenience types are defined, for example
\texttt{Reals} is an alias for \texttt{Read<Real>}.
Likewise, \texttt{LOs} is an alias for \texttt{Read<LO>}.

Consistent with our philosophy of keeping data as much
on the GPU as possible, and taking advantage of the
benefits of immutable data structures, the majority
of data in Omega\_h is stored as a \texttt{Read},
after being created as a \texttt{Write}.
In the rare occasions when data is read or written
to file, or transmitted across a CPU-connected
network interface (e.g. MPI), then the \texttt{HostRead}
and \texttt{HostWrite} classes are used.

\subsection{Maps}

Maps are a conceptual data structure in Omega\_h, although not
a concrete C++ type, because they can simply be represented
as a single array of integers (\texttt{Read<LO>}).
Here we define a map as mapping from one index space to another.
An index space is a range of consecutive non-negative integers,
typically starting at zero and ending at some $(n-1)$, where
$n$ is the cardinality of the index space.
An index space is an implicit representation of a set of $n$ things.

\subsection{Graphs}

Another prevalent data structure in Omega\_h is called a \texttt{Graph},
and is a representation of a directed graph (as defined by graph theory) in ``compressed row"
format, which is optimized for traversing the nodes of a graph, and at each node
$a$ of the graph, querying which nodes $b$ are adjacent to $a$
(equivalently, which edges $(a,b)$ exist in the graph).

The structure consists of two integer arrays (the integer type is \texttt{LO}, the
type used for indexing all arrays).
The first array (\texttt{a2ab}) is a map from source nodes $a$
to edges $(a,b)$ which are sorted by source node.
Because the ``\texttt{ab}" index space is edges sorted by source node,
it follows that all edges $(a,b)$ with the same $a$ have consecutive
indices, and so all that needs to be stored for each $a$ is the
first and last index in the \texttt{ab} space.
In addition, because the last index for one node always precedes
the first index for the next node, only one of them needs to be stored.
As such, the array \texttt{a2ab} has size $(n_a + 1)$, where $n_a$ is the
total number of source nodes.
Given a node in the \texttt{a} index space, the first and last indices
in the \texttt{ab} index space are at \texttt{a2ab[a]} and 
\texttt{a2ab[a + 1]}.
The second array maps the sorted edges to their destination node, and
so is named \texttt{ab2b}.
Here is the typical way in which such a structure is accessed
(in non-parallel host code):

\begin{lstlisting}[language=c++]
for (LO a = 0; a < na; ++a) {
  for (auto ab = a2ab[a]; ab < a2ab[a + 1]; ++ab) {
    auto b = ab2b[ab];
    // do something with the (a,b) pair
  }
}
\end{lstlisting}

The name ``compressed row" comes from a popular format for storing sparse matrices,
and in some sense the \texttt{Graph} data structure is simply that sparse matrix format
with the non-zero values omitted, because the sparsity pattern of a matrix alone is
sufficient to encode a directed graph.

In the special cases where the directed graph has constant degree (i.e. all the
nodes $a$ have the same number of adjacent nodes $b$),
then we omit the \texttt{a2ab} array, because its indices are known to
be multiples of the constant degree.
Downward adjacencies are a common example of this, e.g. we know that all triangles
have exactly three vertices.
In this case, the iteration from triangles to vertices might look like:

\begin{lstlisting}[language=c++]
for (LO tri = 0; tri < ntris; ++tri) {
  for (auto tv = (tri * 3); tv < ((tri + 1) * 3); ++tv) {
    auto vert = tri_verts2verts[tv];
    // do something with the (tri,vert) pair
  }
}
\end{lstlisting}

\subsection{The Mesh: A (Mostly) Immutable Cache}

The Omega\_h \texttt{Mesh} structure represents a triangle or tetrahedral
mesh of a 2D or 3D domain.
The mesh is described in terms of adjacency arrays, which are implemented as
\texttt{Graph} data structures.
These adjacency arrays describe the relationships between simplices (
vertices, edges, triangles, and tetrahedra) in the mesh.
In Omega\_h, a mesh that contains an ``element" (e.g. a tetrahedron)
must also contain all the simplices on its boundary
(e.g. the four triangles, six edges, and four vertices of the tetrahedron).
Elements usually share these boundary simplices, for example to ``adjacent"
tetrahedra share a triangle, three edges, and three vertices.
Although a mesh can be uniquely described by the adjacency from elements
to vertices, storing a full topological representation is
valuable for doing complex, robust mesh adaptation.

In an \texttt{Omega\_h::Mesh} structure, the adjacency relationships
cannot be changed once specified or derived.
The mesh usually begins by specifying the element-to-vertex adjacency,
and Omega\_h will proceed to derive all other relevant adjacencies as
needed.
The only way to change the adjacencies is to create a new \texttt{Mesh}
object.

For efficiency, all adjacency relationships are cached after being derived,
since many of them are expensive to derive and are often requested repeatedly
throughout mesh adaptation and simulation.

\end{document}
